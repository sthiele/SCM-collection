Previously, we covered the sign consistency rule for backward propagation
 which works for transitions between steady states.
In this part I introduce another consistency rule that also works for steady state transitions,
 and increases the predictive power of sign consistency methods.

\section*{Forward propagation}

The backward reasoning implemented by Rule~\ref{rule_bw} is very cautious. 
It expresses the assumption is that for every effect there must be a cause
 but it excludes the assumption that a change must have and effect.
Figure~\ref{fig:weak_prop} shows all labelings $\mu_i$ that are consistent with Rule~\ref{rule_bw} and were $\mu_i(b)=\plus$.
In all labelings it holds that $\mu_i(a)=\plus$, so we can predict very precisely the cause for an increase in $b$.
But the labelings for $c$ are ambiguous, although we see an increase an $b$ we cannot predict an increase in $c$.

\begin{figure}
\centering
  \begin{tikzpicture}[->,semithick,>=stealth',scale=1.2]
    \tikzstyle{node}=[draw=node_gray, circle, minimum size=2.0em, fill=none,text=black]
    \tikzstyle{up}=[style=node,text=black,text opacity=1,fill=node_green,draw=edge_green]
    \tikzstyle{dn}=[style=node,text=black,text opacity=1,fill=node_red,draw=edge_red]
    \tikzstyle{zn}=[style=node,text=black,text opacity=1,fill=node_blue,draw=edge_blue]

    \node[up]       (a) at (0,0)    {$a$};
    \node[up,thick] (b) at (0,-0.8) {$b$};
    \node[zn]       (c) at (0,-1.6) {$c$};
    \node[]         (l) at (0,-2.4) {$\mu_1 $};
    \path
     (0,0.3) edge (a)
     (a) edge[edge_green] (b)
     (b) edge[edge_green] (c)
     ;
  \end{tikzpicture}  
  ~~~~~  
  \begin{tikzpicture}[->,semithick,>=stealth',scale=1.2]
    \tikzstyle{node}=[draw=node_gray, circle, minimum size=2.0em, fill=none,text=black]
    \tikzstyle{up}=[style=node,text=black,text opacity=1,fill=node_green,draw=edge_green]
    \tikzstyle{dn}=[style=node,text=black,text opacity=1,fill=node_red,draw=edge_red]
    \tikzstyle{zn}=[style=node,text=black,text opacity=1,fill=node_blue,draw=edge_blue]

    \node[up]       (a) at (0,0)    {$a$};
    \node[up,thick] (b) at (0,-0.8) {$b$};
    \node[up]       (c) at (0,-1.6) {$c$};
    \node[]         (l) at (0,-2.4) {$\mu_2 $};
    \path
     (0,0.3) edge (a)
     (a) edge[edge_green] (b)
     (b) edge[edge_green] (c)
     ;
  \end{tikzpicture} 

\caption{Prediction under consistency Rule~1. 
  Backward propagation does not allow an unambiguous prediction for $c$.}
\label{fig:weak_prop} 
\end{figure}

To increase the predictive power of our method we define a new consistency rule 
 that implements the assumption that a change in one variable has an effect on its successors.
We define this rule for independent regulations $i \rightarrow j$ were a change in $i$ is sufficient to cause an change in $j$.
In the next part we will introduce cooperative (dependent) regulations  
 were it depends on other variables in the system whether the change in $i$ will have an effect on $j$.
 
\begin{figure}
\begin{tabular}{p{65pt}p{65pt}p{65pt}p{65pt}p{65pt}}
 \centering $S_1$ & \centering $S_2$ & \centering $S_3$ & \centering $S_4$ & \centering $S_5$ %& $S_6$
 \cr &
 \\
  \centering
  \begin{tikzpicture}[->,semithick,>=stealth',scale=1.2]
    \tikzstyle{node}=[draw=node_gray, circle, minimum size=2.0em, fill=none,text=black]
    \tikzstyle{cone}=[draw=black, circle, minimum size=1.0em, fill=none,text=black]

    \node[node] (a)  at (-0.5,0.8)  {\scriptsize $0$};
    \node[node] (b)  at (0.5,0.8)   {\scriptsize $0$};
    \node[node] (c)  at (0,-0.2)    {\scriptsize $0$};
    \path
     (-0.5,1.1) edge[] (a)
     ( 0.5,1.1) edge[] (b)
     (a) edge[edge_green] (c)
     (b) edge[edge_green] (c)
     ;
  \end{tikzpicture}
 &
  \centering
  \begin{tikzpicture}[->,semithick,>=stealth',scale=1.2]
    \tikzstyle{node}=[draw=node_gray, circle, minimum size=2.0em, fill=none,text=black]
    \tikzstyle{cone}=[draw=black, circle, minimum size=1.0em, fill=none,text=black]

    \node[node] (a)  at (-0.5,0.8)  {\scriptsize $0$};
    \node[node] (b)  at (0.5,0.8)   {\scriptsize $5$};
    \node[node] (c)  at (0,-0.2)    {\scriptsize $5$};
    \path
     (-0.5,1.1) edge[] (a)
     ( 0.5,1.1) edge[] (b)
     (a) edge[edge_green] (c)
     (b) edge[edge_green] (c)
     ;
  \end{tikzpicture}
 &
  \centering
  \begin{tikzpicture}[->,semithick,>=stealth',scale=1.2]
    \tikzstyle{node}=[draw=node_gray, circle, minimum size=2.0em, fill=none,text=black]
    \tikzstyle{cone}=[draw=black, circle, minimum size=1.0em, fill=none,text=black]

    \node[node] (a)  at (-0.5,0.8)  {\scriptsize $0$};
    \node[node] (b)  at (0.5,0.8)   {\scriptsize $10$};
    \node[node] (c)  at (0,-0.2)    {\scriptsize $10$};
    \path
     (-0.5,1.1) edge[] (a)
     ( 0.5,1.1) edge[] (b)
     (a) edge[edge_green] (c)
     (b) edge[edge_green] (c)
     ;
  \end{tikzpicture}
 &
  \centering
  \begin{tikzpicture}[->,semithick,>=stealth',scale=1.2]
    \tikzstyle{node}=[draw=node_gray, circle, minimum size=2.0em, fill=none,text=black]
    \tikzstyle{cone}=[draw=black, circle, minimum size=1.0em, fill=none,text=black]

    \node[node] (a)  at (-0.5,0.8)  {\scriptsize $5$};
    \node[node] (b)  at (0.5,0.8)   {\scriptsize $5$};
    \node[node] (c)  at (0,-0.2)    {\scriptsize $10$};
    \path
     (-0.5,1.1) edge[] (a)
     ( 0.5,1.1) edge[] (b)
     (a) edge[edge_green] (c)
     (b) edge[edge_green] (c)
     ;
  \end{tikzpicture}
 &
  \centering
  \begin{tikzpicture}[->,semithick,>=stealth',scale=1.2]
    \tikzstyle{node}=[draw=node_gray, circle, minimum size=2.0em, fill=none,text=black]
    \tikzstyle{cone}=[draw=black, circle, minimum size=1.0em, fill=none,text=black]

    \node[node] (a)  at (-0.5,0.8)  {\scriptsize $5$};
    \node[node] (b)  at (0.5,0.8)   {\scriptsize $10$};
    \node[node] (c)  at (0,-0.2)    {\scriptsize $15$};
    \path
     (-0.5,1.1) edge[] (a)
     ( 0.5,1.1) edge[] (b)
     (a) edge[edge_green] (c)
     (b) edge[edge_green] (c)
     ;
  \end{tikzpicture}
%  &
%   \begin{tikzpicture}[->,semithick,>=stealth',scale=1.2]
%     \tikzstyle{node}=[draw=node_gray, circle, minimum size=2.0em, fill=none,text=black]
%     \tikzstyle{cone}=[draw=black, circle, minimum size=1.0em, fill=none,text=black]
% 
%     \node[node] (a)  at (-0.5,0.8)  {\scriptsize $10$};
%     \node[node] (b)  at (0.5,0.8)   {\scriptsize $10$};  
%     \node[node] (c)  at (0,-0.2)    {\scriptsize $20$};
%     \path
%      (-0.5,1.1) edge[] (a)
%      ( 0.5,1.1) edge[] (b)
%      (a) edge[edge_green] (c)
%      (b) edge[edge_green] (c)
%      ;
%   \end{tikzpicture}
\end{tabular}
\caption{Five quantitative stable states of a variable with two independent regulations.}
\label{fig:or_states}
\end{figure}


\begin{figure}
\centering
\begin{tabular}{cccccc}
 $S_1\rightsquigarrow S_1$ & $S_1\rightsquigarrow S_2 $ & $S_1\rightsquigarrow S_4$
 &  $S_3\rightsquigarrow \tilde{S}_5 $ &  $S_5\rightsquigarrow \tilde{S}_5 $ &  $S_5\rightsquigarrow \tilde{S}_3 $  \\
\\                      
 \begin{tikzpicture}[->,semithick,>=stealth',scale=1.2]
    \tikzstyle{node}=[draw=node_gray, circle, minimum size=2.0em, fill=none,text=black]
    \tikzstyle{up}=[style=node,text=black,text opacity=1,fill=node_green,draw=edge_green]
    \tikzstyle{dn}=[style=node,text=black,text opacity=1,fill=node_red,draw=edge_red]
    \tikzstyle{zn}=[style=node,text=black,text opacity=1,fill=node_blue,draw=edge_blue]
    \tikzstyle{cone}=[draw=black, circle, minimum size=1.0em, fill=none,text=black]

    \node[zn] (a)  at (-0.5,0.8)  {\scriptsize $0$};
    \node[zn] (b)  at (0.5,0.8)   {\scriptsize $0$}; 
    \node[zn] (c)  at (0,-0.2)    {\scriptsize $0$};
    \path
     (-0.5,1.1) edge[] (a)
     ( 0.5,1.1) edge[] (b)
     (a) edge[edge_green] (c)
     (b) edge[edge_green] (c)
     ;
  \end{tikzpicture}
 &
  \begin{tikzpicture}[->,semithick,>=stealth',scale=1.2]
    \tikzstyle{node}=[draw=node_gray, circle, minimum size=2.0em, fill=none,text=black]
    \tikzstyle{up}=[style=node,text=black,text opacity=1,fill=node_green,draw=edge_green]
    \tikzstyle{dn}=[style=node,text=black,text opacity=1,fill=node_red,draw=edge_red]
    \tikzstyle{zn}=[style=node,text=black,text opacity=1,fill=node_blue,draw=edge_blue]
    \tikzstyle{cone}=[draw=black, circle, minimum size=1.0em, fill=none,text=black]

    \node[zn] (a)  at (-0.5,0.8)  {\scriptsize $0$};
    \node[up] (b)  at (0.5,0.8)   {\scriptsize $\plus$};
    \node[up] (c)  at (0,-0.2)    {\scriptsize $\plus$};
    \path
     (-0.5,1.1) edge[] (a)
     ( 0.5,1.1) edge[] (b)
     (a) edge[edge_green] (c)
     (b) edge[edge_green] (c)
     ;
  \end{tikzpicture}
 &
  \begin{tikzpicture}[->,semithick,>=stealth',scale=1.2]
    \tikzstyle{node}=[draw=node_gray, circle, minimum size=2.0em, fill=none,text=black]
    \tikzstyle{up}=[style=node,text=black,text opacity=1,fill=node_green,draw=edge_green]
    \tikzstyle{dn}=[style=node,text=black,text opacity=1,fill=node_red,draw=edge_red]
    \tikzstyle{zn}=[style=node,text=black,text opacity=1,fill=node_blue,draw=edge_blue]
    \tikzstyle{cone}=[draw=black, circle, minimum size=1.0em, fill=none,text=black]

    \node[up] (a)  at (-0.5,0.8)  {\scriptsize $\plus$};
    \node[up] (b)  at (0.5,0.8)   {\scriptsize $\plus$};
    \node[up] (c)  at (0,-0.2)    {\scriptsize $\plus$};
    \path
     (-0.5,1.1) edge[] (a)
     ( 0.5,1.1) edge[] (b)
     (a) edge[edge_green] (c)
     (b) edge[edge_green] (c)
     ;
  \end{tikzpicture}
 &
  \begin{tikzpicture}[->,semithick,>=stealth',scale=1.2]
    \tikzstyle{node}=[draw=node_gray, circle, minimum size=2.0em, fill=none,text=black]
    \tikzstyle{up}=[style=node,text=black,text opacity=1,fill=node_green,draw=edge_green]
    \tikzstyle{dn}=[style=node,text=black,text opacity=1,fill=node_red,draw=edge_red]
    \tikzstyle{zn}=[style=node,text=black,text opacity=1,fill=node_blue,draw=edge_blue]
    \tikzstyle{am}=[style=node,text=black,text opacity=1,fill=node_yell,draw=edge_blue]
    \tikzstyle{cone}=[draw=black, circle, minimum size=1.0em, fill=none,text=black]


    \node[up] (a)  at (-0.5,0.8)  {\scriptsize $\plus$};
    \node[dn] (b)  at (0.5,0.8)   {\scriptsize $\minus$};  
    \node[dn] (c)  at (0,-0.2)    {\scriptsize $\minus$};
%     \node[label] (cl)  at (0.0,-0.5)  {\tiny $\minus~0~\plus$};
    \path
     (-0.5,1.1) edge[] (a)
     ( 0.5,1.1) edge[] (b)
     (a) edge[edge_green] (c)
     (b) edge[edge_green] (c)
     ;
  \end{tikzpicture} &
  \begin{tikzpicture}[->,semithick,>=stealth',scale=1.2]
    \tikzstyle{node}=[draw=node_gray, circle, minimum size=2.0em, fill=none,text=black]
    \tikzstyle{up}=[style=node,text=black,text opacity=1,fill=node_green,draw=edge_green]
    \tikzstyle{dn}=[style=node,text=black,text opacity=1,fill=node_red,draw=edge_red]
    \tikzstyle{zn}=[style=node,text=black,text opacity=1,fill=node_blue,draw=edge_blue]
%     \tikzstyle{am}=[style=node,text=black,text opacity=1,fill=node_yell,draw=edge_blue]
    \tikzstyle{cone}=[draw=black, circle, minimum size=1.0em, fill=none,text=black]


    \node[up] (a)  at (-0.5,0.8)  {\scriptsize $\plus$};
    \node[dn] (b)  at (0.5,0.8)   {\scriptsize $\minus$};
    \node[zn] (c)  at (0,-0.2)    {\scriptsize 0};
    \path
     (-0.5,1.1) edge[] (a)
     ( 0.5,1.1) edge[] (b)
     (a) edge[edge_green] (c)
     (b) edge[edge_green] (c)
     ;
  \end{tikzpicture} &
  \begin{tikzpicture}[->,semithick,>=stealth',scale=1.2]
    \tikzstyle{node}=[draw=node_gray, circle, minimum size=2.0em, fill=none,text=black]
    \tikzstyle{up}=[style=node,text=black,text opacity=1,fill=node_green,draw=edge_green]
    \tikzstyle{dn}=[style=node,text=black,text opacity=1,fill=node_red,draw=edge_red]
    \tikzstyle{zn}=[style=node,text=black,text opacity=1,fill=node_blue,draw=edge_blue]
    \tikzstyle{am}=[style=node,text=black,text opacity=1,fill=node_yell,draw=edge_blue]
    \tikzstyle{cone}=[draw=black, circle, minimum size=1.0em, fill=none,text=black]


    \node[up] (a)  at (-0.5,0.8)  {\scriptsize $\plus$};
    \node[dn] (b)  at (0.5,0.8)   {\scriptsize $\minus$};
    \node[up] (c)  at (0,-0.2)    {\scriptsize $\plus$};
    \path
     (-0.5,1.1) edge[] (a)
     ( 0.5,1.1) edge[] (b)
     (a) edge[edge_green] (c)
     (b) edge[edge_green] (c)
     ;
  \end{tikzpicture}
 \\  
 \\
 a) & b) & c) & d) & e) & f)
\end{tabular}
\caption{Sign labeling representation of possible transitions among the states in Figure~\ref{fig:or_states}.
The $\sim$ means that the values of the input nodes have been switched left to right in the state.}
\label{fig:or_shifts}
\end{figure}

Figure~\ref{fig:or_states} shows exemplarily five quantitative representations of stable states of 
 a system were one variable is independently regulated by two predecessors.
In Figure~\ref{fig:or_shifts} we see the possible sign labelings representing different state transitions.
We see that for the cases $a$-$c$ there exists an unambiguous input output behavior.
In the cases $d$-$f$ we see different outputs for the same input.
This behavior is implemented by the following forward propagation rule.
%
\begin{srule}\label{rule_fw}{\bf (forward propagation)} 0-change can only occur in a variable that does not depend on other variables with change, 
 or if it receives opposing regulations.

Let $(V,E,\sigma)$ be an IG.
Then a labeling $\mu : V \rightarrow \{\plus,\minus,0\}$ satisfies Rule~\ref{rule_fw} for node $i \in V$
 iff
 \begin{itemize}
  \item $\mu(i) \neq 0$, or
  \item there is no edge $j \rightarrow i$ in~$E$ such that $\mu(j)\sigma(j,i) \in \{\plus,\minus\}$, or
  \item there exist at least two edges $j_1 \rightarrow i$ and $j_2 \rightarrow i$ in~$E$
 such that $\mu(j_1)\sigma(j_1,i) + \mu(j_2)\sigma(j_2,i) = 0$. 
 \end{itemize}
\end{srule}

This rule implements forward propagation by restricting the occurrence of $0$.
In other words Rule~\ref{rule_fw} limits the cases were a change in an upstream node can have no effect
 on the regulated down stream node.
We can see now that labeling $\mu_1$ in Figure~\ref{fig:weak_prop} does not satisfy Rule~\ref{rule_fw} for $c$.
Only labeling $\mu_2$ satisfies both Rule~\ref{rule_bw} and~\ref{rule_fw} for all nodes, 
 leaving us with only one possible behavior.
 
 
\section*{Conclusion} 

In this part I have introduced the forward propagation rule for independent regulations, and
we have seen how forward propagation increases the predictive power of sign consistency methods.
In the next part I will show how we can implement forward propagation for cooperative (dependent) regulations.
