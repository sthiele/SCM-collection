In the previous parts we covered inputs and how we can reason on perturbations of the system.
In this part we will see how we can relate actual experimental measurements to our model.

\section*{Suitable experiments}

First, we have to clarify what kind of measurements can be used for our approach and the sign consistency rules we apply.
Because the sign consistency methods reasons over the changes between states,
 we need the measurements of at least two states.
% This means we deal with at least two measurements a reference experiment and an operating experiment.
Further, the measured system states have to be compatible with the sign consistency rules that we want to apply.
For example, Rule~1 reasons over steady state changes. 
This means, if we want to use Rule~1 we have to ensure that the measured states are steady states.
There exist other sign consistency rules that work under different assumptions and reason over transitional states or early responses of the system,
 in any case it must always be ensured that the measured changes satisfy the assumptions.

 
\section*{Discretization} 

If the experimental data satisfies our assumptions, we can discretize the data and compute sign constraints.
For measured variables $v \in V$,
 we compute the difference between the values measured in a reference state and an operational state $obs(v) = (v_R - v_O)$.
The result is a value in $\mathbb{R}$. 
Depending on four thresholds $t_1 \leq t_2 < 0 < t_3 \leq t_4$, that should be defined on a case by case basis,
 we can use the quantitative change to define five discrete observation types.
As illustrated in Figure~\ref{fig:sign_discretization}, 
these thresholds define a (partial) mapping $\nu: V \rightarrow \{ \minus, \uncertaindecrease, 0, \uncertainincrease, \plus \}$ as follows:
\[
\nu(s)=\left \{
\begin{array}{lll}
    \minus              & |\quad        & obs(s) \leq t_1, \\  
    \uncertaindecrease  & |\quad t_1 <  & obs(s) \leq t_2,\\
    0                   & |\quad t_2 <  & obs(s) < t_3, \\
    \uncertainincrease  & |\quad t_3 \leq & obs(s) < t_4,\\
    \plus               & |\quad t_4 \leq & obs(s) \\
    \bot                & |\quad else.
\end{array} \right . 
\]
%
  \begin{figure}[h!]
  \centering
  \begin{tikzpicture}[scale=1.0]

  \tikzstyle{label}=[text=black]
  \tikzstyle{species}=[minimum size=0.2ex, maximum size=0.2ex, fill=none,text=black]
  
  \shade[left color=edge_red, right color=node_red](0,0) rectangle (1,0.5);
  \shade[left color=node_red, right color=node_blue](1,0) rectangle (4,0.5);
  \shade[left color=node_blue,right color=node_green](4,0) rectangle (7,0.5);
  \shade[left color=node_green,right color=edge_green](7,0) rectangle (8,0.5);
  
  \draw[<->,semithick]  (0,0) -- (8,0);
  \draw[dashed] (1.3,-0.1) -- (1.3,0.5);
  \draw[dashed] (3.3,-0.1) -- (3.3,0.5);
  \draw[dashed] (4.7,-0.1) -- (4.7,0.5);
  \draw[dashed] (6.8,-0.1) -- (6.8,0.5);
  \node[label]  (l1) at (1.3,-0.3)  {$t_1$};
  \node[label]  (l2) at (3.3,-0.3)  {$t_2$};
  \node[label]  (l3) at (4.7,-0.3)  {$t_3$};
  \node[label]  (l4) at (6.8,-0.3)  {$t_4$};
  \node[label]  (l5) at (4,-0.3)  {$0$};
  
  \node[label]  (l1) at (0.6,1)  {~$\minus$};
  \node[label]  (l2) at (2.3,1)  {~$\uncertaindecrease$};
  \node[label]  (l3) at (4,1)  {$0$};
  \node[label]  (l4) at (5.7,1)  {~$\uncertainincrease$};
  \node[label]  (l5) at (7.4,1)  {$\plus$};
  
  \draw [decorate,decoration={brace,amplitude=5pt,raise=2pt},yshift=0pt]
  (0,0.5) -- (1.3,0.5) node [black,midway,xshift=0.8cm] {\footnotesize };
  
  \draw [decorate,decoration={brace,amplitude=5pt,raise=2pt},yshift=0pt]
  (1.3,0.5) -- (3.3,0.5) node [black,midway,xshift=0.8cm] {\footnotesize };

  \draw [decorate,decoration={brace,amplitude=5pt,raise=2pt},yshift=0pt]
  (3.3,0.5) -- (4.7,0.5) node [black,midway,xshift=0.8cm] {\footnotesize };

  \draw [decorate,decoration={brace,amplitude=5pt,raise=2pt},yshift=0pt]
  (4.7,0.5) -- (6.8,0.5) node [black,midway,xshift=0.8cm] {\footnotesize };

  \draw [decorate,decoration={brace,amplitude=5pt,raise=2pt},yshift=0pt]
  (6.8,0.5) -- (8,0.5) node [black,midway,xshift=0.8cm] {\footnotesize };
\end{tikzpicture}
  \caption{Discretization of observed changes into sign constraints.}
  \label{fig:sign_discretization}
  \end{figure} 

We consider measurements which are smaller than $t_1$, bigger than $t_4$, and between $t_2$ and $t_3$
 as certain (decrease \minus, increase \plus, no-change 0)
 while measurements that are between $t_1$ and $t_2$ (resp. $t_3$ and $t_4$) are uncertain
 (uncertain-decrease $\uncertaindecrease$, uncertain-increase $\uncertainincrease$) and not exactly classifiable. 

\section*{Consistency checking model wrt. observed changes}
 
Given an IG $(V,E,\sigma)$ 
 and observed changes $\nu$ 
 we can define the rules that restrict the possible sign labelings of the IG model s.t. they reflect the observed behavior.
% For this purpose we look for labelings $\mu: V \rightarrow \{ \minus, 0, \plus \}$ 
% that satisfy the rules defined below.
It is important to notice that we are looking for labelings $\mu_i: V \rightarrow \{ \minus, 0, \plus \}$
 using the \emph{three} sign labels $\{\minus,0,\plus\}$
 whereas $\nu$ defines an observation labeling
 based on the \emph{five} observation labels  $\{ \minus, \uncertaindecrease, 0, \uncertainincrease, \plus \}$
 representing the discretized measurements.

\begin{srule}\label{def:constraint_obs}{\bf (satisfy observations)}
Every observed change must be consistent with the labeling.

Let $(V,E,\sigma)$ be an IG, $\nu$ an observation labeling $\nu: V \rightarrow \{ \minus, \uncertaindecrease, 0, \uncertainincrease, \plus \}$
 representing observed changes in the system.
Then a labeling $\mu : V \rightarrow \{\plus,\minus,0\}$ satisfies Rule~\ref{def:constraint_obs} for node $i\in V$
 iff
 \begin{itemize}
  \item $\mu(i)=\plus$ and $\nu(i)\in \{ \plus, \uncertainincrease \}$, or
  \item $\mu(i)=0$ and $\nu(i)\in \{ \uncertainincrease, 0, \uncertaindecrease\}$, or
  \item $\mu(i)=\minus$ and $\nu(i) \in \{ \uncertaindecrease, \minus \}$, or
  \item $\nu(i)= \bot$.  
 \end{itemize}

\end{srule}
%
This rule ensures that a labeling is compatible with the measured changes.
Note, uncertain measurements restrict the labeling of a node to two out of the three values $\{\plus, \minus, 0\}$,
while measurements with high certainty fix a node's label to exactly one value.
In Figure~\ref{fig:obs_label} we see a partial observation labeling and the corresponding sign labelings 
 that satisfy Rule~\ref{def:constraint_obs}.
Note that only the labeling $\mu_{11}$ satisfies Rule~1.
This means using sign consistency we can use incomplete and uncertain measurement data to make predictions on the behavior of unobserved system variables.

\begin{figure}
\begin{tabular}{ccccc}%{p{70pt}p{60pt}p{60pt}p{60pt}p{60pt}}
  \begin{tikzpicture}[->,semithick,>=stealth',scale=1.2]
    \tikzstyle{node}=[draw=node_gray, circle, minimum size=2.0em, fill=none,text=black]
    \tikzstyle{up}=[style=node,text=black,text opacity=1,fill=node_green,draw=edge_green]
    \tikzstyle{dn}=[style=node,text=black,text opacity=1,fill=node_red,draw=edge_red]
    \tikzstyle{zn}=[style=node,text=black,text opacity=1,fill=node_blue,draw=edge_blue]

    \node[node] (a) at (0,0)        {\scriptsize $\plus$};
    \node[node] (b) at (-0.0,-1.0)  {\scriptsize $\minus$};
    \node[node] (c) at (0.8 ,-0.5)  {\scriptsize $0$};
    \node[node] (d) at (-0.8,-0.5)  {$\uncertaindecrease$};
    \node[node] (e) at (-0.8,-1.5)  {$\uncertainincrease$};
    \node[node] (f) at (0.0,-2.0)   {$\bot$};
    \node[]     (l) at (0,-2.7)     {$\nu$};
    \path
     (0.0,0.3) edge (a)
     (a) edge[edge_red, -|] (b) 
     (a) edge[edge_green] (c)
     (a) edge[edge_green] (d)
     (b) edge[edge_green] (c)
     (b) edge[edge_green] (d)
     (d) edge[edge_green] (e)
     (e) edge[edge_red, -|,bend right=50] (f)
     (f) edge[edge_red, -|,bend right=50] (e);
  \end{tikzpicture}
  &
  \centering
  \begin{tikzpicture}[->,semithick,>=stealth',scale=1.2]
    \tikzstyle{node}=[draw=node_gray, circle, minimum size=2.0em, fill=none,text=black]
    \tikzstyle{up}=[style=node,text=black,text opacity=1,fill=node_green,draw=edge_green]
    \tikzstyle{dn}=[style=node,text=black,text opacity=1,fill=node_red,draw=edge_red]
    \tikzstyle{zn}=[style=node,text=black,text opacity=1,fill=node_blue,draw=edge_blue]

    \node[up] (a) at (0,0)        {\scriptsize $\plus$};
    \node[dn] (b) at (-0.0,-1.0)  {\scriptsize $\minus$};
    \node[zn] (c) at (0.8 ,-0.5)  {\scriptsize $0$};
    \node[dn] (d) at (-0.8,-0.5)  {\scriptsize $\minus$};
    \node[zn] (e) at (-0.8,-1.5)  {\scriptsize $0$};
    \node[dn] (f) at (0.0,-2.0)   {\scriptsize $\minus$};
    \node[]     (l) at (0,-2.7)     {$\mu_1$};
    \path
     (0.0,0.3) edge (a)
     (a) edge[edge_red, -|] (b) 
     (a) edge[edge_green] (c)
     (a) edge[edge_green] (d)
     (b) edge[edge_green] (c)
     (b) edge[edge_green] (d)
     (d) edge[edge_green] (e)
     (e) edge[edge_red, -|,bend right=50] (f)
     (f) edge[edge_red, -|,bend right=50] (e);
  \end{tikzpicture}
  &
  \centering
  \begin{tikzpicture}[->,semithick,>=stealth',scale=1.2]
    \tikzstyle{node}=[draw=node_gray, circle, minimum size=2.0em, fill=none,text=black]
    \tikzstyle{up}=[style=node,text=black,text opacity=1,fill=node_green,draw=edge_green]
    \tikzstyle{dn}=[style=node,text=black,text opacity=1,fill=node_red,draw=edge_red]
    \tikzstyle{zn}=[style=node,text=black,text opacity=1,fill=node_blue,draw=edge_blue]

    \node[up] (a) at (0,0)        {\scriptsize $\plus$};
    \node[dn] (b) at (-0.0,-1.0)  {\scriptsize $\minus$};
    \node[zn] (c) at (0.8 ,-0.5)  {\scriptsize $0$};
    \node[zn] (d) at (-0.8,-0.5)  {\scriptsize $0$};
    \node[zn] (e) at (-0.8,-1.5)  {\scriptsize $0$};
    \node[dn] (f) at (0.0,-2.0)   {\scriptsize $\minus$};
    \node[]     (l) at (0,-2.7)     {$\mu_1$};
    \path
     (0.0,0.3) edge (a)
     (a) edge[edge_red, -|] (b) 
     (a) edge[edge_green] (c)
     (a) edge[edge_green] (d)
     (b) edge[edge_green] (c)
     (b) edge[edge_green] (d)
     (d) edge[edge_green] (e)
     (e) edge[edge_red, -|,bend right=50] (f)
     (f) edge[edge_red, -|,bend right=50] (e);
  \end{tikzpicture}
  &
  \centering
  \begin{tikzpicture}[->,semithick,>=stealth',scale=1.2]
    \tikzstyle{node}=[draw=node_gray, circle, minimum size=2.0em, fill=none,text=black]
    \tikzstyle{up}=[style=node,text=black,text opacity=1,fill=node_green,draw=edge_green]
    \tikzstyle{dn}=[style=node,text=black,text opacity=1,fill=node_red,draw=edge_red]
    \tikzstyle{zn}=[style=node,text=black,text opacity=1,fill=node_blue,draw=edge_blue]

    \node[up] (a) at (0,0)        {\scriptsize $\plus$};
    \node[dn] (b) at (-0.0,-1.0)  {\scriptsize $\minus$};
    \node[zn] (c) at (0.8 ,-0.5)  {\scriptsize $0$};
    \node[dn] (d) at (-0.8,-0.5)  {\scriptsize $\minus$};
    \node[up] (e) at (-0.8,-1.5)  {\scriptsize $\plus$};
    \node[dn] (f) at (0.0,-2.0)   {\scriptsize $\minus$};
    \node[]     (l) at (0,-2.7)     {$\mu_1$};
    \path
     (0.0,0.3) edge (a)
     (a) edge[edge_red, -|] (b) 
     (a) edge[edge_green] (c)
     (a) edge[edge_green] (d)
     (b) edge[edge_green] (c)
     (b) edge[edge_green] (d)
     (d) edge[edge_green] (e)
     (e) edge[edge_red, -|,bend right=50] (f)
     (f) edge[edge_red, -|,bend right=50] (e);
  \end{tikzpicture}
  &
  \centering
  \begin{tikzpicture}[->,semithick,>=stealth',scale=1.2]
    \tikzstyle{node}=[draw=node_gray, circle, minimum size=2.0em, fill=none,text=black]
    \tikzstyle{up}=[style=node,text=black,text opacity=1,fill=node_green,draw=edge_green]
    \tikzstyle{dn}=[style=node,text=black,text opacity=1,fill=node_red,draw=edge_red]
    \tikzstyle{zn}=[style=node,text=black,text opacity=1,fill=node_blue,draw=edge_blue]

    \node[up] (a) at (0,0)        {\scriptsize $\plus$};
    \node[dn] (b) at (-0.0,-1.0)  {\scriptsize $\minus$};
    \node[zn] (c) at (0.8 ,-0.5)  {\scriptsize $0$};
    \node[zn] (d) at (-0.8,-0.5)  {\scriptsize $0$};
    \node[up] (e) at (-0.8,-1.5)  {\scriptsize $\plus$};
    \node[dn] (f) at (0.0,-2.0)   {\scriptsize $\minus$};
    \node[]     (l) at (0,-2.7)     {$\mu_1$};
    \path
     (0.0,0.3) edge (a)
     (a) edge[edge_red, -|] (b) 
     (a) edge[edge_green] (c)
     (a) edge[edge_green] (d)
     (b) edge[edge_green] (c)
     (b) edge[edge_green] (d)
     (d) edge[edge_green] (e)
     (e) edge[edge_red, -|,bend right=50] (f)
     (f) edge[edge_red, -|,bend right=50] (e);
  \end{tikzpicture}
  \cr &
  \\
  ~ &
  \centering
  \centering
  \begin{tikzpicture}[->,semithick,>=stealth',scale=1.2]
    \tikzstyle{node}=[draw=node_gray, circle, minimum size=2.0em, fill=none,text=black]
    \tikzstyle{up}=[style=node,text=black,text opacity=1,fill=node_green,draw=edge_green]
    \tikzstyle{dn}=[style=node,text=black,text opacity=1,fill=node_red,draw=edge_red]
    \tikzstyle{zn}=[style=node,text=black,text opacity=1,fill=node_blue,draw=edge_blue]

    \node[up] (a) at (0,0)        {\scriptsize $\plus$};
    \node[dn] (b) at (-0.0,-1.0)  {\scriptsize $\minus$};
    \node[zn] (c) at (0.8 ,-0.5)  {\scriptsize $0$};
    \node[dn] (d) at (-0.8,-0.5)  {\scriptsize $\minus$};
    \node[zn] (e) at (-0.8,-1.5)  {\scriptsize $0$};
    \node[zn] (f) at (0.0,-2.0)   {\scriptsize $0$};
    \node[]     (l) at (0,-2.7)     {$\mu_1$};
    \path
     (0.0,0.3) edge (a)
     (a) edge[edge_red, -|] (b) 
     (a) edge[edge_green] (c)
     (a) edge[edge_green] (d)
     (b) edge[edge_green] (c)
     (b) edge[edge_green] (d)
     (d) edge[edge_green] (e)
     (e) edge[edge_red, -|,bend right=50] (f)
     (f) edge[edge_red, -|,bend right=50] (e);
  \end{tikzpicture}
  &
  \centering
  \begin{tikzpicture}[->,semithick,>=stealth',scale=1.2]
    \tikzstyle{node}=[draw=node_gray, circle, minimum size=2.0em, fill=none,text=black]
    \tikzstyle{up}=[style=node,text=black,text opacity=1,fill=node_green,draw=edge_green]
    \tikzstyle{dn}=[style=node,text=black,text opacity=1,fill=node_red,draw=edge_red]
    \tikzstyle{zn}=[style=node,text=black,text opacity=1,fill=node_blue,draw=edge_blue]

    \node[up] (a) at (0,0)        {\scriptsize $\plus$};
    \node[dn] (b) at (-0.0,-1.0)  {\scriptsize $\minus$};
    \node[zn] (c) at (0.8 ,-0.5)  {\scriptsize $0$};
    \node[zn] (d) at (-0.8,-0.5)  {\scriptsize $0$};
    \node[zn] (e) at (-0.8,-1.5)  {\scriptsize $0$};
    \node[zn] (f) at (0.0,-2.0)   {\scriptsize $0$};
    \node[]     (l) at (0,-2.7)     {$\mu_1$};
    \path
     (0.0,0.3) edge (a)
     (a) edge[edge_red, -|] (b) 
     (a) edge[edge_green] (c)
     (a) edge[edge_green] (d)
     (b) edge[edge_green] (c)
     (b) edge[edge_green] (d)
     (d) edge[edge_green] (e)
     (e) edge[edge_red, -|,bend right=50] (f)
     (f) edge[edge_red, -|,bend right=50] (e);
  \end{tikzpicture}
  &
  \centering
  \begin{tikzpicture}[->,semithick,>=stealth',scale=1.2]
    \tikzstyle{node}=[draw=node_gray, circle, minimum size=2.0em, fill=none,text=black]
    \tikzstyle{up}=[style=node,text=black,text opacity=1,fill=node_green,draw=edge_green]
    \tikzstyle{dn}=[style=node,text=black,text opacity=1,fill=node_red,draw=edge_red]
    \tikzstyle{zn}=[style=node,text=black,text opacity=1,fill=node_blue,draw=edge_blue]

    \node[up] (a) at (0,0)        {\scriptsize $\plus$};
    \node[dn] (b) at (-0.0,-1.0)  {\scriptsize $\minus$};
    \node[zn] (c) at (0.8 ,-0.5)  {\scriptsize $0$};
    \node[dn] (d) at (-0.8,-0.5)  {\scriptsize $\minus$};
    \node[up] (e) at (-0.8,-1.5)  {\scriptsize $\plus$};
    \node[zn] (f) at (0.0,-2.0)   {\scriptsize $0$};
    \node[]     (l) at (0,-2.7)     {$\mu_1$};
    \path
     (0.0,0.3) edge (a)
     (a) edge[edge_red, -|] (b) 
     (a) edge[edge_green] (c)
     (a) edge[edge_green] (d)
     (b) edge[edge_green] (c)
     (b) edge[edge_green] (d)
     (d) edge[edge_green] (e)
     (e) edge[edge_red, -|,bend right=50] (f)
     (f) edge[edge_red, -|,bend right=50] (e);
  \end{tikzpicture}
  &
  \centering
  \begin{tikzpicture}[->,semithick,>=stealth',scale=1.2]
    \tikzstyle{node}=[draw=node_gray, circle, minimum size=2.0em, fill=none,text=black]
    \tikzstyle{up}=[style=node,text=black,text opacity=1,fill=node_green,draw=edge_green]
    \tikzstyle{dn}=[style=node,text=black,text opacity=1,fill=node_red,draw=edge_red]
    \tikzstyle{zn}=[style=node,text=black,text opacity=1,fill=node_blue,draw=edge_blue]

    \node[up] (a) at (0,0)        {\scriptsize $\plus$};
    \node[dn] (b) at (-0.0,-1.0)  {\scriptsize $\minus$};
    \node[zn] (c) at (0.8 ,-0.5)  {\scriptsize $0$};
    \node[zn] (d) at (-0.8,-0.5)  {\scriptsize $0$};
    \node[up] (e) at (-0.8,-1.5)  {\scriptsize $\plus$};
    \node[zn] (f) at (0.0,-2.0)   {\scriptsize $0$};
    \node[]     (l) at (0,-2.7)     {$\mu_1$};
    \path
     (0.0,0.3) edge (a)
     (a) edge[edge_red, -|] (b) 
     (a) edge[edge_green] (c)
     (a) edge[edge_green] (d)
     (b) edge[edge_green] (c)
     (b) edge[edge_green] (d)
     (d) edge[edge_green] (e)
     (e) edge[edge_red, -|,bend right=50] (f)
     (f) edge[edge_red, -|,bend right=50] (e);
  \end{tikzpicture}
  \cr &
  \\
   ~ &
  \centering
  \centering
  \begin{tikzpicture}[->,semithick,>=stealth',scale=1.2]
    \tikzstyle{node}=[draw=node_gray, circle, minimum size=2.0em, fill=none,text=black]
    \tikzstyle{up}=[style=node,text=black,text opacity=1,fill=node_green,draw=edge_green]
    \tikzstyle{dn}=[style=node,text=black,text opacity=1,fill=node_red,draw=edge_red]
    \tikzstyle{zn}=[style=node,text=black,text opacity=1,fill=node_blue,draw=edge_blue]

    \node[up] (a) at (0,0)        {\scriptsize $\plus$};
    \node[dn] (b) at (-0.0,-1.0)  {\scriptsize $\minus$};
    \node[zn] (c) at (0.8 ,-0.5)  {\scriptsize $0$};
    \node[dn] (d) at (-0.8,-0.5)  {\scriptsize $\minus$};
    \node[zn] (e) at (-0.8,-1.5)  {\scriptsize $0$};
    \node[up] (f) at (0.0,-2.0)   {\scriptsize $\plus$};
    \node[]     (l) at (0,-2.7)     {$\mu_1$};
    \path
     (0.0,0.3) edge (a)
     (a) edge[edge_red, -|] (b) 
     (a) edge[edge_green] (c)
     (a) edge[edge_green] (d)
     (b) edge[edge_green] (c)
     (b) edge[edge_green] (d)
     (d) edge[edge_green] (e)
     (e) edge[edge_red, -|,bend right=50] (f)
     (f) edge[edge_red, -|,bend right=50] (e);
  \end{tikzpicture}
  &
  \centering
  \begin{tikzpicture}[->,semithick,>=stealth',scale=1.2]
    \tikzstyle{node}=[draw=node_gray, circle, minimum size=2.0em, fill=none,text=black]
    \tikzstyle{up}=[style=node,text=black,text opacity=1,fill=node_green,draw=edge_green]
    \tikzstyle{dn}=[style=node,text=black,text opacity=1,fill=node_red,draw=edge_red]
    \tikzstyle{zn}=[style=node,text=black,text opacity=1,fill=node_blue,draw=edge_blue]

    \node[up] (a) at (0,0)        {\scriptsize $\plus$};
    \node[dn] (b) at (-0.0,-1.0)  {\scriptsize $\minus$};
    \node[zn] (c) at (0.8 ,-0.5)  {\scriptsize $0$};
    \node[zn] (d) at (-0.8,-0.5)  {\scriptsize $0$};
    \node[zn] (e) at (-0.8,-1.5)  {\scriptsize $0$};
    \node[up] (f) at (0.0,-2.0)   {\scriptsize $\plus$};
    \node[]     (l) at (0,-2.7)     {$\mu_1$};
    \path
     (0.0,0.3) edge (a)
     (a) edge[edge_red, -|] (b) 
     (a) edge[edge_green] (c)
     (a) edge[edge_green] (d)
     (b) edge[edge_green] (c)
     (b) edge[edge_green] (d)
     (d) edge[edge_green] (e)
     (e) edge[edge_red, -|,bend right=50] (f)
     (f) edge[edge_red, -|,bend right=50] (e);
  \end{tikzpicture}
  &
  \centering
  \begin{tikzpicture}[->,semithick,>=stealth',scale=1.2]
    \tikzstyle{node}=[draw=node_gray, circle, minimum size=2.0em, fill=none,text=black]
    \tikzstyle{up}=[style=node,text=black,text opacity=1,fill=node_green,draw=edge_green]
    \tikzstyle{dn}=[style=node,text=black,text opacity=1,fill=node_red,draw=edge_red]
    \tikzstyle{zn}=[style=node,text=black,text opacity=1,fill=node_blue,draw=edge_blue]

    \node[up] (a) at (0,0)        {\scriptsize $\plus$};
    \node[dn] (b) at (-0.0,-1.0)  {\scriptsize $\minus$};
    \node[zn] (c) at (0.8 ,-0.5)  {\scriptsize $0$};
    \node[dn] (d) at (-0.8,-0.5)  {\scriptsize $\minus$};
    \node[up] (e) at (-0.8,-1.5)  {\scriptsize $\plus$};
    \node[up] (f) at (0.0,-2.0)   {\scriptsize $\plus$};
    \node[]     (l) at (0,-2.7)     {$\mu_1$};
    \path
     (0.0,0.3) edge (a)
     (a) edge[edge_red, -|] (b) 
     (a) edge[edge_green] (c)
     (a) edge[edge_green] (d)
     (b) edge[edge_green] (c)
     (b) edge[edge_green] (d)
     (d) edge[edge_green] (e)
     (e) edge[edge_red, -|,bend right=50] (f)
     (f) edge[edge_red, -|,bend right=50] (e);
  \end{tikzpicture}
  &
  \centering
  \begin{tikzpicture}[->,semithick,>=stealth',scale=1.2]
    \tikzstyle{node}=[draw=node_gray, circle, minimum size=2.0em, fill=none,text=black]
    \tikzstyle{up}=[style=node,text=black,text opacity=1,fill=node_green,draw=edge_green]
    \tikzstyle{dn}=[style=node,text=black,text opacity=1,fill=node_red,draw=edge_red]
    \tikzstyle{zn}=[style=node,text=black,text opacity=1,fill=node_blue,draw=edge_blue]

    \node[up] (a) at (0,0)        {\scriptsize $\plus$};
    \node[dn] (b) at (-0.0,-1.0)  {\scriptsize $\minus$};
    \node[zn] (c) at (0.8 ,-0.5)  {\scriptsize $0$};
    \node[zn] (d) at (-0.8,-0.5)  {\scriptsize $0$};
    \node[up] (e) at (-0.8,-1.5)  {\scriptsize $\plus$};
    \node[up] (f) at (0.0,-2.0)   {\scriptsize $\plus$};
    \node[]     (l) at (0,-2.7)     {$\mu_1$};
    \path
     (0.0,0.3) edge (a)
     (a) edge[edge_red, -|] (b) 
     (a) edge[edge_green] (c)
     (a) edge[edge_green] (d)
     (b) edge[edge_green] (c)
     (b) edge[edge_green] (d)
     (d) edge[edge_green] (e)
     (e) edge[edge_red, -|,bend right=50] (f)
     (f) edge[edge_red, -|,bend right=50] (e);
  \end{tikzpicture}
\end{tabular}
  \caption{A labeling $\nu$ of observed changes and the corresponding sign labelings $\mu_i$ satisfying Rule~\ref{def:constraint_obs}.}
  \label{fig:obs_label}
\end{figure}

If for a given IG and a set of observed changes $\nu$ there exist no sign labeling that satisfies Rule~\ref{def:constraint_obs} then
the model is not able to explain the observed behavior. 
In other words either the model or the measurements are faulty.
In this case we can apply a variety of repair methods can be applied to repair model or data.
These methods will be explained in detail following parts.



\section*{Conclusion}
In this part I explained which kind of experimental measurements can be used in combination with sign consistency methods.
I presented a method to discretize quantitative measurements into sign constraints, and
 showed how one can handle incompleteness and uncertainty in measurements.
In the next part I will introduce a new sign consistency rule that implements forward propagation to
 increase the predictive power of sign consistency methods.
