In this part I will introduce a new global consistency rule to ensure that every change
is justified by a chain of influences that can be traced back to an input node.
This natural constraint is especially useful to exclude self-justification of changes via positive feedback loops.


\section*{Circular regulation}

In most biological systems we find circular regulation via feedback loops.
For the consistency rules that we've covered so far circular regulations pose the problem that
 they allow to represent state transistions for which we can not identify the reason that trigger the state change.
For example in Figure~\ref{fig:founded} we see an IG with labelings that explain an increase in $d$.
Both total labelings satisfy the propagation rules (Rule~\ref{rule_bw} and~\ref{rule_fw}).
For $\mu_1$ we see a circular up regulation in $b$ and $c$, which is used as explanation for the increase in $d$,
 but we don't know why $b$ and $c$ have increase in the first place.
Only the labeling $\mu_{2}$ allows us to identify a trigger for the state change, i.e. the increase in $a$.

\begin{figure}
\small
\centering
\begin{tabular}{p{70pt}p{70pt}p{70pt}p{70pt}}
  \begin{tikzpicture}[->,semithick,>=stealth',scale=1.2]
    \tikzstyle{label}=[text=black] 
    \tikzstyle{node}=[draw=node_gray, circle, minimum size=2.0em, fill=none,text=black]
    \tikzstyle{up}=[style=node,text=black,text opacity=1,fill=node_green,draw=edge_green]
    \tikzstyle{dn}=[style=node,text=black,text opacity=1,fill=node_red,draw=edge_red]
    \tikzstyle{zn}=[style=node,text=black,text opacity=1,fill=node_blue,draw=edge_blue]
    \node[zn] (A) at (0.5,1)    {$a$};
    \node[up] (B) at (1.5,1)    {$c$};
    \node[up] (C) at (1.5,1.75) {$b$};
    \node[up,thick] (D) at (1,0)      {$d$};
    \node[label] (l1) at (1,-0.7) {$\mu_1$};
    
    \path[every node/.style={anchor=south}]
     (0.5,1.3) edge[] (A)
     (A) edge[edge_green] (D)
     (B) edge[edge_green,bend right=50] (C.east)
     (C) edge[edge_green,bend right=50] (B.west)     
     (B) edge[edge_green] (D);
  \end{tikzpicture}
%   &
%   \begin{tikzpicture}[->,semithick,>=stealth',scale=1.2]
%     \tikzstyle{label}=[text=black] 
%     \tikzstyle{node}=[draw=node_gray, circle, minimum size=2.0em, fill=none,text=black]
%     \tikzstyle{up}=[style=node,text=black,text opacity=1,fill=node_green,draw=edge_green]
%     \tikzstyle{dn}=[style=node,text=black,text opacity=1,fill=node_red,draw=edge_red]
%     \tikzstyle{zn}=[style=node,text=black,text opacity=1,fill=node_blue,draw=edge_blue]
%     \node[up] (A) at (0.5,1)    {$a$};
%     \node[dn] (B) at (1.5,1)    {$c$};
%     \node[dn] (C) at (1.5,1.75) {$b$};
%     \node[up,thick] (D) at (1,0)      {$d$};   
%     \node[label] (l1) at (1,-0.7) {$\mu_2$};
%     
%     \path[every node/.style={anchor=south}]
%      (0.5,1.3) edge[] (A)
%      (A) edge[edge_green] (D)
%      (B) edge[edge_green,bend right=50] (C.east)
%      (C) edge[edge_green,bend right=50] (B.west)     
%      (B) edge[edge_green] (D);
%   \end{tikzpicture}
  & 
  \begin{tikzpicture}[->,semithick,>=stealth',scale=1.2]
    \tikzstyle{label}=[text=black] 
    \tikzstyle{node}=[draw=node_gray, circle, minimum size=2.0em, fill=none,text=black]
    \tikzstyle{up}=[style=node,text=black,text opacity=1,fill=node_green,draw=edge_green]
    \tikzstyle{dn}=[style=node,text=black,text opacity=1,fill=node_red,draw=edge_red]
    \tikzstyle{zn}=[style=node,text=black,text opacity=1,fill=node_blue,draw=edge_blue]

    \node[up] (A) at (0.5,1)    {$a$};
    \node[zn] (B) at (1.5,1)    {$c$};
    \node[zn] (C) at (1.5,1.75) {$b$};
    \node[up,thick] (D) at (1,0)      {$d$};
    \node[label] (l1) at (1,-0.7) {$\mu_2$};
    
    \path[every node/.style={anchor=south}]
     (0.5,1.3) edge[] (A)
     (A) edge[edge_green] (D)
     (B) edge[edge_green,bend right=50] (C.east)
     (C) edge[edge_green,bend right=50] (B.west)     
     (B) edge[edge_green] (D);
  \end{tikzpicture}
%   &
%   \begin{tikzpicture}[->,semithick,>=stealth',scale=1.2]
%     \tikzstyle{label}=[text=black] 
%     \tikzstyle{node}=[draw=node_gray, circle, minimum size=2.0em, fill=none,text=black]
%     \tikzstyle{up}=[style=node,text=black,text opacity=1,fill=node_green,draw=edge_green]
%     \tikzstyle{dn}=[style=node,text=black,text opacity=1,fill=node_red,draw=edge_red]
%     \tikzstyle{zn}=[style=node,text=black,text opacity=1,fill=node_blue,draw=edge_blue]
% 
%     \node[up] (A) at (0.5,1)    {$a$};
%     \node[up] (B) at (1.5,1)    {$c$};
%     \node[up] (C) at (1.5,1.75) {$b$};
%     \node[up,thick] (D) at (1,0)      {$d$};
%     \node[label] (l1) at (1,-0.7) {$\mu_4$};
%     
%     \path[every node/.style={anchor=south}]
%      (0.5,1.3) edge[] (A)
%      (A) edge[edge_green] (D)
%      (B) edge[edge_green,bend right=50] (C.east)
%      (C) edge[edge_green,bend right=50] (B.west)     
%      (B) edge[edge_green] (D);
%   \end{tikzpicture}
  \end{tabular}  
  \caption{Example for an influence graph with self-regulation. An increase in $d$ can be 
   explained either by self-activation in $b$ and $c$ or by the input node $a$.}
  \label{fig:founded}
\end{figure} 

To filter labelings which represent circular explanations we introduce the following consistency rule.
  
\begin{srule}\label{rule_founded}{\bf (a change must be founded in an input).}
Let $(V,E,\sigma)$ be an IG and $I \subseteq V$ the input nodes.
Then a labeling $\mu : V \rightarrow \{\plus,\minus,0\}$ satisfies Rule~\ref{rule_founded} for $i \in V$ if
 \begin{itemize}
  \item $i \in I$, or
  \item $\mu(i)=0$, or
  \item there exist a path $(v_0,\dots,v_k)$ in $E$ with $v_0 \in I$, $v_k=i$ and
 $\mu(v_{n-1})\sigma(v_{n-1},v_n)=\mu(v_{n})$ for all $n=1\dots k$.
 \end{itemize}
\end{srule}

Using Rule~\ref{rule_founded} we can avoid manual removal of positive
 feedback loops as done in other approaches, and identify state transitions which can be explained by external perturbations.
Only the labeling $\mu_2$ satisfies Rule~\ref{rule_founded}.
Consistency rule~\ref{rule_founded} is especially useful if we want apply sign consistency methods in the context of perturbation experiments.
When we are actually interested in the response to perturbations,
or if we want to identify possible perturbations that trigger desired state transitions.

\section*{Conclusion}

This part introduced a consistency rule that allows us to exclude unfounded self-regulations.
In the next part I will show how we can relate our model to actual measurement data via sign constraints.
